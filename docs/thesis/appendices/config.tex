\chapter{Konfigurációs változók}
\label{appx:config}

\vspace*{-1.5\baselineskip}

\begin{table}[H]
	\centering
	\begin{tabular}{ | m{0.25\textwidth} | m{0.25\textwidth} | m{0.45\textwidth} | }
		\hline
		\textbf{Konfigurációs változó} & \textbf{Alap érték} & \textbf{Megjegyzés} \\
		\hline \hline
		
		\multicolumn{3}{|l|}{backend} \\
		\hline
		\small{IWA\_CorsAllowUrls} & http://127.0.0.1:8100 & Vesszővel elválasztva az engedélyezett publikus frontend url-ek felsorolása. Pl.: \url{https://andipeter.me} \\ \hline
		\small{IWA\_SeedInitialData} & "true" & "true"-nál teszt adatokkal tölti fel az adatbázist, amúgy "false" \\ \hline
		\small{IWA\_MYSQL\_HOST} & db & MariaDB adatbázis host neve \\ \hline
		\small{IWA\_MYSQL\_HOST} & db & MariaDB host neve \\ \hline
		\small{IWA\_MYSQL\_PORT} & 3306 & MariaDB portja \\ \hline
		\small{IWA\_MYSQL\_DB} & iwa & MariaDB-n belüli adatbázis \\ \hline
		\small{IWA\_MYSQL\_USER} & root & MariaDB felhasználója \\ \hline
		\small{IWA\_MYSQL\_PASS} & kebab & MariaDB felhasználójának jelszava \\ \hline
		\small{volumes} & \tiny{./avatars:/app/AvatarData} & A profilképek perzisztens tárolása a lokális avatars mappában  \\ \hline
		\small{ports} & 5000:80 & A konténer 80-as portja a hoston az 5000-es portra forwardolása \\
		\hline
	\end{tabular}
	\caption{Konfigurációs változók beállításai 1}
	\label{tab:config}
\end{table}

\begin{table}[H]
	\centering
	\begin{tabular}{ | m{0.25\textwidth} | m{0.25\textwidth} | m{0.45\textwidth} | }
		\hline
		\textbf{Konfigurációs változó} & \textbf{Alap érték} & \textbf{Megjegyzés} \\
		\hline \hline
		\multicolumn{3}{|l|}{db} \\
		\hline
		\tiny{MYSQL\_ROOT\_PASSWORD} & kebab & MariaDB root jelszava \\
		\hline
		\small{volumes} & \tiny{./db\_data:/var/lib/mysql} & MariaDB adatbázis perzisztens tárolása a lokális db\_data mappában  \\ \hline
		\small{ports} & 3306:3306 & A konténer 3306-os portja a hoston a 3306-os portra forwardolása \\ \hline
		\hline
		\multicolumn{3}{|l|}{frontend} \\
		\hline
		\small{API\_URL} & http://127.0.0.1:5000 & A backend publikus elérési url-je. Pl.: \url{https://andipeter.me}/api/ \\
		\hline
		\small{ports} & 8100:80 & A konténer 80-as portja a hoston az 8100-es portra forwardolása \\ \hline
				
	\end{tabular}
	\caption{Konfigurációs változók beállításai 2}
	\label{tab:config2}
\end{table}