{\Huge{Témabejelentő}}

A szakdolgozat célja egy időpont foglaló webes alkalmazás létrehozása. Az alkalmazásban vállalkozók (pl.: edzők, magán tanárok) szabad időpontokat hirdethetnek, melyeket ügyfeleik lefoglalhatnak. Ez az alkalmazás lehetővé teszi, hogy az egyéni-, kis- és középvállalkozók egyszerűen tudják egyeztetni ügyfeleikkel a munkáikat. Továbbá, a szoftver számon tartja a múltbeli foglaltidőpontokat, melyek így lekérdezhetők, így például a vállalkozó számlázás során egyszerűen meg tudja állapítani, hogy az adott hónapra hány alkalmat vett igénybe egy kliens.

A program két különálló részből áll, egy webes frontendből, amit Javascript-el és hasonló modern technológiákkal valósítok meg és egy backend API-ból melyet C\# ASP.NET-ben kivitelezek. A frontend a backenddel http requestekkel kommunikál, a backend pedig egy adatbázist használ az adatok tárolására. A dolgozatomban rámutatok ennek az architektúrának az előnyeire és hátrányaira egy monolitikus MVC alapú webes alkalmazással szemben.

\todo{URL-ekhez mindenhol last visited dátum}
