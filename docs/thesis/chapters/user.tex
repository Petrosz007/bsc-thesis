\chapter{Felhasználói dokumentáció} % User guide
\label{ch:user}

\section{Rendszerkövetelmények}
Szerver oldalon: 64 bit-es Windows 10 vagy Linux operációs rendszer, 2GB RAM, legalább 5GB tárhely az adatoknak, port nyitási lehetőség, domain cím, esetleg SSL tanusítvány.

Kliens oldalon: Legalább Chrome 90, Firefox 88, Edge 90, ezek mind asztali számítógépen, legalább 1280x720-as képernyő felbontással.

\section{Telepítés}

Az alkalmazást legegyszerűbben Docker\footnote{\url{https://www.docker.com/} (utolsó elérés: 2021.05.10)} segítségével lehet telepíteni. Van lehetőség Docker nélkül is, viszont az több konfigurációval és üzemeltetési idővel és költséggel jár.

\subsection{Telepítés Dockerrel}

A Dockeres telepítéshez szükséges a Docker\footnote{\url{https://www.docker.com/get-started} (utolsó elérés: 2021.05.10)} és Docker Compose\footnote{\url{https://docs.docker.com/compose/install/} (utolsó elérés: 2021.05.10)} telepítése.

A fő mappában a \textit{docker-compse.yml} fájlban találhatók meg a konténerek konfigurációi. Három konténerből áll, egy MariaDB \footnote{\url{https://mariadb.org/} (utolsó elérés: 2021.05.10)} adatbázisból, a backend REST API-ból és a frontendből.
A yml fájlban a konfigurációs lehetőségek az \ref{tab:config} táblázatban találhatók.

Az alap beállításokkal a \textit{http://localhost:8100}-on érhető el az alkalmazás. HTTPS-t és tűzfalat érdemes bekonfigurálni egy Reverse Proxy\cite{nginxReverseProxy}-val, például Nginx-el.

Az alkalmazást ez után a \textit{docker-compose up} paranccsal indíthatjuk el. Első futtatásra ez eltarthat pár percig, mert a Dockernek le kell töltenie a megfelelő alap konténereket az internetről és utána létre kell hozni ezeket a konténereket a forráskódból.

További docker-compose parancsok a docker-compose dokumentációjában\footnote{\url{https://docs.docker.com/compose/reference/} (utolsó elérés: 2021.05.10)} találhatók.

\subsection{Telepítés Docker nélkül}

Az alkalmazás futtatásához szükség lesz egy MariaDB szerverre és azon belül egy \textit{iwa} nevű adatbázisra. Az alkalmazás Linux és Windows rendszereken is futhat, ehhez a megfelelő backend fájl futtatása szükséges.

A backend-hez szükséges a .NET ASP.NET Core 5.0 runtime\footnote{\url{https://dotnet.microsoft.com/download/dotnet/5.0} (utolsó elérés: 2021.05.14)} telepítése. Futtatása parancssorból az \textit{IWA\_Backend.API} futtatható fájllal lehet. A konfigurációja az \text{appsettings.json} fájlban található, kitöltése az \ref{tab:config} táblázat alapján történik. A backend így a host 80-as portján fog futni. \\ Ezt a \textit{$--$urls=http://localhost:5001/} konzoli argumentummal lehet megváltoztatni, ebben az esetben az 5001-es porton futna az alkalmazás.

A frontend statikus HTML, JS és CSS fájlokból áll, ezt például Apache\footnote{\url{https://httpd.apache.org/} (utolsó elérés: 2021.05.10)} vagy Nginx\footnote{\url{https://www.nginx.com/} (utolsó elérés: 2021.05.10)} szerverekkel, vagy más hasonló webhost szolgáltatásokkal lehet kitelepíteni. A frontend konfigurációja a mappájában a \textit{config.js} fájlban történik, kitöltési útmutató az \ref{tab:config} táblázatban található.

\section{Funkciók leírása}
\label{sec:functionDefinitions}

Az alkalmazásban lehet regisztrálni ügyfélként vagy vállalkozóként. Az oldalt lehet bejelentkezve vagy bejelentkezés nélkül böngészni.

\subsubsection{Kategóriák}

A vállalkozók létrehozhatnak kategóriákat. A kategória effektíve egy időpont típus, például személyi edzés. A kategóriák megegyszerűsítik az új időpontok létrehozását, mert a különböző időpontok közti azonos adatokat enkapszulálják, az időpontnál így csak az időpont specifikus adatokat kell megadni. Egy kategóriának lehet egy leírása, ára, ajánlott max résztvevő száma és láthatósága. Az ajánlott max résztvevőszám azt jelenti, hogy egy új időpont létrehozásánál alapból ez a szám lesz a max résztvevők mezőben, viszont ettől el lehet térni időpontról időpontra, például egy csoportos edzésre a Margit szigeten többen jöhetnek mint a Hősök tereire.

Egy kategória láthatósága a következőt jelenti. Ha nyílt egy esemény, akkor bárki láthatja, bárki jelentkezhet rá. Ha egy esemény nem nyílt, akkor csak azok az emberek láthatják és jelentkezhetnek rá, akik engedélyezett résztvevőként fel lettek véve a kategóriára. Ennek az a szerepe, hogy például egy Családi edzésre hétvégén ne tudjon mindenki jelentkezni, csak az előre felvett családtagok. Vagy például egy kedvezményes árazású időpontnak más lehet a kategóriája.

Kategóriákat nem lehet törölni, abból az okból, hogy akkor az összes hozzá tartozó időpont is törlődne, ezzel múltbeli időpontok adatai elvesznének.

\subsubsection{Időpontok}

Új időpont hirdetésénél az időponthoz kell választani egy kategóriát, kezdő és vég időpontot. Opcionálisan meg lehet változtatni a max résztvevő számot. Van lehetőség alapból felvenni ügyfeleket az időpontra, például ha a vállalkozó már előre leegyeztetett egy időpontot de még nem írta ki az alkalmazáson, akkor az ügyfélnek nem kell bejelentkeznie és lefoglalni az időpontot.

Időpont szerkesztésnél a vállalkozónak van lehetősége változtatni egy időpont összes értékén. A kategórián például azért változtathat, mert Angol óra helyett Német órát tartott az ügyfélnek, vagy Páros edzés helyett Személyi Edzést, mert közbe jött valami. Lehet az időpontra jelentkezett felhasználókat is módosítani, lejelentkeztetni és felvenni ügyfeleket, akár az időpont után is. Például valaki lemondott egy edzést és beugrott helyette valaki más, a nap végén pedig így helyesen tudja adminisztrálni ezt a vállalkozó.

\subsubsection{Számlázás}

A számlázás funkciónál a vállalkozók adott felhasználók lefoglalt időpontjaiból tudnak számlát generálni egy időszakra, például Április 1 és 30 között. Ez a számla jelenlegi formájában nem minősül NAV által elfogadott számlának, viszont a vállalkozónak nagyon jó segítség, hogy a saját számlázó szoftverébe (pl.: számlázz.hu) miről írjon számlát. Az alkalmazásba azért se került online fizetési lehetőség vagy számlázz.hu integráció, mert a valóságban az időpontokon kívül mást is tartalmazni szokott a számla (pl.: edzőterem bérlet, edzésterv) és ezekre akkor ezen felül egy külön számlát kéne kiállítania a vállalkozónak.

\clearpage

\section{Használat}

A regisztrációs oldalon lehet regisztrálni az alkalmazásba. Itt meg kell adni egy felhasználónevet, saját nevet, email címet, jelszót. Van lehetőség vállalkozóként regisztrálni, ekkor meg kell adni a foglalkozást és egy magáról szóló rövid leírást is.

Regisztráció után automatikusan be is jelentkeztet az oldal. Ekkor elérhető a saját profil oldal, ahova a navigációs sáv jobb fölső sarkában a saját névre kattintva lehet eljutni.

A profil oldalon láthatók a személyes adatok, a Szerkesztés gombbal lehet ezeket szerkeszteni. Vállalkozók a Profilkép Frissítése gombbal lecserélhetik a profilképüket.

\begin{figure}[H]
    \centering
    \subfigure[Profil szerkesztése]{
        \includegraphics[width=0.45\textwidth]{css/profile_editor}
    }
    \subfigure[Profilkép frissítése]{
        \includegraphics[width=0.45\textwidth]{css/avatar_editor}
    }
    \caption{Profil és profilkép szerkesztése}
\end{figure}

% \clearpage

Kijelentkezni szintén a profil oldalon lehet. Bejelentkezni a Navigációs sávon a Bejelentkezés gombbal lehet, felhasználónévvel és jelszóval.

% \begin{figure}[H]
%     \centering
%     \includegraphics[width=1.0\textwidth]{css/register}
%     \caption{Regisztrációs oldal}
% \end{figure}

% \begin{figure}[H]
%     \centering
%     \includegraphics[width=1.0\textwidth]{css/login}
%     \caption{Bejelentkező oldal}
% \end{figure}

\subsection{Ügyfeleknek}

Az alkalmazásban a Vállalkozók fülre kattintva lehet böngészni a vállalkozók között. Itt megjelennek a vállalkozók profilképei, nevei, foglalkozásai és leírásai.

\begin{figure}[H]
    \centering
    \includegraphics[width=1.0\textwidth]{css/contractors}
    \caption{Vállalkozó böngésző oldal}
\end{figure}

Egy vállalkozó nevére kattintva böngészhetjük a vállalkozó időpontjait. Ezeket lehet szűrni kategóriák és kezdő dátum szerint, például szűrhetünk csak május 15 és június 13 közötti időpontokra. Egy időpontnál látható a kezdés és alatta a befejezés időpontja, az időpont kategóriájának neve, leírása, ára és, hogy hány szabad hely van. Bejelentkezéssel látható egy Foglalás vagy Lemondás gomb, ezzel lehet lefoglalni vagy lemondani az időpontot. A vállalkozói oldalon nem jelennek meg olyan időpontok, amik tele vannak vagy amikre az ügyfél nincs engedélyezve.

\begin{figure}[H]
    \centering
    \includegraphics[width=1.0\textwidth]{css/contractor}
    \caption{Egy vállalkozó oldala}
\end{figure}

A Foglalások oldalon találhatók az általunk lefoglalt időpontok. Ugyan úgy lehet őket szűrni, mint a vállalkozók időpontjait.

\begin{figure}[H]
    \centering
    \includegraphics[width=1.0\textwidth]{css/booked}
    \caption{Lefoglalt időpontok oldala}
\end{figure}

\clearpage

\subsection{Vállalkozóknak}

A vállalkozóknak elérhető az összes funkció, ami az ügyfeleknek. A Vállalkozói oldalon tudják a kategóriáikat és az időpontjaikat kezelni. A Kategóriák alatt találhatók meg a kategóriáik, ezeket meg tudják tekinteni és szerkeszteni. Az Időpontok alatt szűrhetően a saját időpontok találhatók.

\begin{figure}[H]
    \centering
    \includegraphics[width=1.0\textwidth]{css/own_everything}
    \caption{Saját kategóriák és időpontok}
\end{figure}

\clearpage

\subsubsection{Kategóriák és Időpontok kezelése}

Egy kategória létrehozásánál vagy szerkesztésénél felvehetik egy kategória adatait. Ha nem nyílt az esemény, akkor egy lenyíló menüből választhatnak az ügyfelek közül, akiket a Felvétel gombbal engedélyezhetnek egy kategóriára.

\begin{figure}[H]
    \centering
    \subfigure[Kategória megtekintése]{
        \includegraphics[width=0.45\textwidth]{css/cat_viewer}
    }
    \subfigure[Kategória frissítése]{
        \includegraphics[width=0.45\textwidth]{css/cat_editor}
    }
    \caption{Kategória megtekintése és szerkesztése}
\end{figure}

Az időpontokat ugyan úgy lehet szűrni mint a vállalkozói oldalon, azon felül egy megtekintő, szerkesztő és törlő gomb található rajtuk. A megtekintés és a szerkesztés hasonló egy kategória szerkesztéséhez és megtekintéséhez.

\clearpage

\subsubsection{Számlázás}

A Számlázás oldalon lehet adott ügyfélnek egy időintervallumban a lefoglalt időpontjaikat összegezni. A számla letöltése gombbal a jelenlegi szűrés eredményeiből lehet egy PDF formátumú számlát generálni.

\begin{figure}[H]
    \centering
    \includegraphics[width=1.0\textwidth]{css/reports2}
    \caption{Számlázás oldal}
\end{figure}

\begin{figure}[H]
    \centering
    \includegraphics[width=1.0\textwidth]{css/invoice}
    \caption{PDF formátumú számla}
\end{figure}

\subsubsection{Hibakezelés}

Az alkalmazásban a különböző hibaüzeneteket felugró ablakokkal jelenítem meg. Ezek a hibák mindig leírják, hogy mi a probléma, ebből a felhasználó tud következtetni, hogy mit rontott el. Például nem megfelelő formátumú fájlt válaszott ki profilképnek vagy túl nagy a fájl mérete.

A hibaüzeneteket a jobb fölső sarkukban az X gombbal lehet bezárni.

\begin{figure}[H]
    \centering
    \includegraphics[width=1.0\textwidth]{css/errors}
    \caption{Felugró hibaüzenetekre példa}
\end{figure}















% Lorem ipsum dolor sit amet $\mathbb{N}$\nomenclature{$\mathbb{N}$}{Set of natural numbers}, consectetur adipiscing elit. Duis nibh leo, dapibus in elementum nec, aliquet id sem. Suspendisse potenti. Nullam sit amet consectetur nibh. Donec scelerisque varius turpis at tincidunt. Cras a diam in mauris viverra vehicula. Vivamus mi odio, fermentum vel arcu efficitur, lacinia viverra nibh. Aliquam aliquam ante mi, vel pretium arcu dapibus eu. Nulla finibus ante vel arcu tincidunt, ut consectetur ligula finibus. Mauris mollis lectus sed ipsum bibendum, ac ultrices erat dictum. Suspendisse faucibus euismod lacinia $\mathbb{Z}$\nomenclature{$\mathbb{Z}$}{Set of integer numbers}.


% \section{Felsorolások} % Enumerations and lists

% Etiam vel odio ante. Etiam pulvinar nibh quis massa auctor congue. Pellentesque quis odio vitae sapien molestie vestibulum sit amet et quam. Pellentesque vel dui eget enim hendrerit finibus at sit amet libero. Quisque sollicitudin ultrices enim, nec porta magna imperdiet vitae. Cras condimentum nunc dui, eget molestie nunc accumsan vel.

% \begin{itemize}
% 	\item Fusce in aliquet neque, in pretium sem.
% 	\item Donec tincidunt tellus id lectus pretium fringilla.
% 	\item Nunc faucibus, erat pretium tempus tempor, tortor mi fringilla neque, ac congue ex dui vitae mauris.
% \end{itemize}

% Donec dapibus sodales ante, at scelerisque nunc laoreet sit amet. Mauris porttitor tincidunt neque, vel ullamcorper neque pulvinar et. Integer eu lorem euismod, faucibus lectus sed, accumsan felis. Nunc ornare mi at augue vulputate, eu venenatis magna mollis. Nunc sed posuere dui, et varius nulla. Sed mollis nibh augue, eget scelerisque eros ornare nec.

% \begin{enumerate}
% 	\item\label{step:first} Donec pretium et quam a cursus. Ut sollicitudin tempus urna et mollis.
% 	\item Aliquam et aliquam turpis, sed fermentum mauris. Nulla eget ex diam.
% 	\item Donec eget tellus pharetra, semper neque eget, rutrum diam Step~\ref{step:first}.
% \end{enumerate}

% Praesent porta, metus eget eleifend consequat, eros ligula eleifend ex, a pellentesque mi est vitae urna. Vivamus turpis nunc, iaculis non leo eget, mattis vulputate tellus. Maecenas rutrum eros sem, pharetra interdum nulla porttitor sit amet. In vitae viverra ante. Maecenas sit amet placerat orci, sed tincidunt velit. Vivamus mattis, enim vel suscipit elementum, quam odio venenatis elit\footnote{Phasellus faucibus varius purus, nec tristique enim porta vitae.}, et mollis nulla nunc a risus. Praesent purus magna, tristique sed lacus sit amet, convallis malesuada magna. 

% \begin{description}
% 	\item[Vestibulum venenatis] malesuada enim, ac auctor erat vestibulum et. Phasellus id purus a leo suscipit accumsan.
% 	\item[Orci varius natoque] penatibus et magnis dis parturient montes, nascetur ridiculus mus. Nullam interdum rhoncus nisl, vel pharetra arcu euismod sagittis. Vestibulum ac turpis auctor, viverra turpis at, tempus tellus.
% 	\item[Morbi dignissim] erat ut rutrum aliquet. Nulla eu rutrum urna. Integer non urna at mauris scelerisque rutrum sed non turpis.
% \end{description}

% \subsection{Szoros térközű felsorolások} % Lists with narrow spacing inbetween items

% Phasellus ultricies, sapien sit amet ultricies placerat, velit purus viverra ligula, id consequat ipsum odio imperdiet enim:
% \begin{compactenum}
% 	\item Maecenas eget lobortis leo.
% 	\item Donec eget libero enim.
% 	\item In eu eros a eros lacinia maximus ullamcorper eget augue.
% \end{compactenum}

% \bigskip

% In quis turpis metus. Proin maximus nibh et massa eleifend, a feugiat augue porta. Sed eget est purus. Duis in placerat leo. Donec pharetra eros nec enim convallis:
% \begin{compactitem}
% 	\item Pellentesque odio lacus.
% 	\item Maximus ut nisl auctor.
% 	\item Sagittis vulputate lorem.
% 	\item \todo{Hmm ide kéne még valami}
% \end{compactitem}

% \bigskip

% Vestibulum ante ipsum primis in faucibus orci luctus et ultrices posuere cubilia Curae; Sed lorem libero, dignissim vitae gravida a, ornare vitae est.
% \begin{compactdesc}
% 	\item[Cras maximus] massa commodo pellentesque viverra.
% 	\item[Morbi sit] amet ante risus. Aliquam nec sollicitudin mauris
% 	\item[Ut aliquam rhoncus sapien] luctus viverra arcu iaculis posuere
% \end{compactdesc}


% \section{Képek, ábrák} % Images and figures

% Aliquam vehicula luctus mi a pretium. Nulla quam neque, maximus nec velit in, aliquam mollis tortor. Aliquam erat volutpat. Curabitur vitae laoreet turpis. Integer id diam ligula. Nulla sodales purus id mi consequat, eu venenatis odio pharetra. Cras a arcu quam. Suspendisse augue risus, pulvinar a turpis et, commodo aliquet turpis. Nulla aliquam scelerisque mi eget pharetra. Mauris sed posuere elit, ac lobortis metus. Proin lacinia sit amet diam sed auctor. Nam viverra orci id sapien sollicitudin, a aliquam lacus suscipit, Figure~\ref{fig:example-1}:

% \begin{figure}[H]
% 	\centering
% 	\includegraphics[width=0.6\textwidth,height=100px]{elte_cimer_szines}
% 	\caption{Quisque ac tincidunt leo}
% 	\label{fig:example-1}
% \end{figure}

% \subsection{Képek szegélyezése} % Framing figures

% Ut aliquet nec neque eget fermentum. Cras volutpat tellus sed placerat elementum. Quisque neque dui, consectetur nec finibus eget, blandit id purus. Nam eget ipsum non nunc placerat interdum.

% \begin{figure}[H]
% 	\centering
% 	\includegraphics[width=0.6\textwidth,height=100px,frame]{elte_cimer_szines}
% 	\caption{Quisque ac tincidunt leo}
% \end{figure}

% \subsection{Képek csoportosítása} % Subfigures

% In non ipsum fermentum urna feugiat rutrum a at odio. Pellentesque habitant morbi tristique senectus et netus et malesuada fames ac turpis egestas. Nulla tincidunt mattis nisl id suscipit. Sed bibendum ac felis sed volutpat. Nam pharetra nisi nec facilisis faucibus. Aenean tristique nec libero non commodo. Nulla egestas laoreet tempus. Nunc eu aliquet nulla, quis vehicula dui. Proin ac risus sodales, gravida nisi vitae, efficitur neque, Figure~\ref{fig:example-2}:

% \begin{figure}[H]
% 	\centering
% 	\subfigure[Vestibulum quis mattis urna]{
% 		\includegraphics[width=0.45\linewidth]{elte_cimer_szines}}
% 	\hspace{5pt}
% 	\subfigure[Donec hendrerit quis dui sit amet venenatis]{
% 		\includegraphics[width=0.45\linewidth]{elte_cimer_szines}}
% 	\caption{Aenean porttitor mi volutpat massa gravida}
% 	\label{fig:example-2}
% \end{figure}

% Nam et nunc eget elit tincidunt sollicitudin. Quisque ligula ipsum, tempor vitae tortor ut, commodo rhoncus diam. Pellentesque habitant morbi tristique senectus et netus et malesuada fames ac turpis egestas. Phasellus vehicula quam dui, eu convallis metus porta ac.


% \section{Táblázatok} % Tables

% Nam magna ex, euismod nec interdum sed, sagittis nec leo. Nam blandit massa bibendum mattis tristique. Phasellus tortor ligula, sodales a consectetur vitae, placerat vitae dolor. Aenean consequat in quam ac mollis. 

% \begin{table}[H]
% 	\centering
% 	\begin{tabular}{ | m{0.25\textwidth} | m{0.65\textwidth} | }
% 		\hline
% 		\textbf{Phasellus tortor} & \textbf{Aenean consequat} \\
% 		\hline \hline
% 		\emph{Sed malesuada} & Aliquam aliquam velit in convallis ultrices. \\
% 		\hline
% 		\emph{Purus sagittis} &  Quisque lobortis eros vitae urna lacinia euismod. \\
% 		\hline
% 		\emph{Pellentesque} & Curabitur ac lacus pellentesque, eleifend sem ut, placerat enim. Ut auctor tempor odio ut dapibus. \\
% 		\hline
% 	\end{tabular}
% 	\caption{Maecenas tincidunt non justo quis accumsan}
% 	\label{tab:example-1}
% \end{table}

% \subsection{Sorok és oszlopok egyesítése} % Multi rows and multi columns

% Mauris a dapibus lectus. Vestibulum commodo nibh ante, ut maximus magna eleifend vel. Integer vehicula elit non lacus lacinia, vitae porttitor dolor ultrices. Vivamus gravida faucibus efficitur. Ut non erat quis arcu vehicula lacinia. Nulla felis mauris, laoreet sed malesuada in, euismod et lacus. Aenean at finibus ipsum. Pellentesque dignissim elit sit amet lacus congue vulputate.

% \begin{table}[htb]
% 	\centering
% 	\begin{tabular}{ | c | r | r | r | r | r | r | }
% 		\hline
% 		\multirow{2}{*}{\textbf{Quisque}} & \multicolumn{2}{ c | }{\textbf{Suspendisse}} & \multicolumn{2}{ c | }{\textbf{Aliquam}} & \multicolumn{2}{ c | }{\textbf{Vivamus}} \\
% 		\cline{2-7}
% 		& Proin & Nunc & Proin & Nunc & Proin & Nunc \\
% 		\hline \hline		
% 		Leo & 2,80 MB & 100\% & 232 KB & 8,09\% & 248 KB & 8,64\% \\
% 		\hline
% 		Vel & 9,60 MB & 100\% & 564 KB & 5,74\% & 292 KB & 2,97\% \\
% 		\hline
% 		Auge & 78,2 MB & 100\% & 52,3 MB & 66,88\% & 3,22 MB & 4,12\% \\
% 		\hline 
% 	\end{tabular}
% 	\caption[Rövid cím a táblázatjegyzékbe]{Vivamus ac arcu fringilla, fermentum neque sed, interdum erat. Mauris bibendum mauris vitae enim mollis, et eleifend turpis aliquet.}
% 	\label{tab:example-2}
% \end{table}

% \subsection{Több oldalra átnyúló táblázatok} % Long tables over multiple pages

% Nunc porta placerat leo, sit amet porttitor dui porta molestie. Aliquam at fermentum mi. Maecenas vitae lorem at leo tincidunt volutpat at nec tortor. Vivamus semper lacus eu diam laoreet congue. Vivamus in ipsum risus. Nulla ullamcorper finibus mauris non aliquet. Vivamus elementum rhoncus ex ut porttitor.

% \begin{center}
% 	\begin{longtable}{ | p{0.3\textwidth} | p{0.7\textwidth} | }
		
% 		\hline
% 		\multicolumn{2}{|c|}{\textbf{Praesent aliquam mauris enim}}
% 		\\ \hline
		
% 		\emph{Suspendisse potenti} & \emph{Lorem ipsum dolor sit amet}
% 		\\ \hline \hline
% 		\endfirsthead % első oldal fejléce
		
% 		\hline
% 		\emph{Suspendisse potenti} & \emph{Lorem ipsum dolor sit amet}
% 		\\ \hline \hline
% 		\endhead % többi oldal fejléce
		
% 		\hline
% 		\endfoot % többi oldal lábléce
		
% 		\endlastfoot % utolsó oldal lábléce
		
% 		\emph{Praesent}
% 		& Nulla ultrices et libero sit amet fringilla. Nunc scelerisque ante tempus sapien placerat convallis.
% 		\\ \hline
		
% 		\emph{Luctus}
% 		& Integer hendrerit erat massa, non hendrerit risus convallis at. Curabitur ultrices, justo in imperdiet condimentum, neque tortor luctus enim, luctus posuere massa erat vitae nibh.
% 		\\ \hline
		
% 		\emph{Egestas}
% 		& Duis fermentum feugiat augue in blandit. Mauris a tempor felis. Pellentesque ultricies tristique dignissim. Pellentesque aliquam semper tristique. Nam nec egestas dolor. Vestibulum id elit quis enim fringilla tempor eu a mauris. Aliquam vitae lacus tellus. Phasellus mauris lectus, aliquam id leo eget, auctor dapibus magna. Fusce lacinia felis ac elit luctus luctus.
% 		\\ \hline
		
% 		\emph{Dignissim}
% 		& Praesent aliquam mauris enim, vestibulum posuere massa facilisis in. Suspendisse potenti. Nam quam purus, rutrum eu augue ut, varius vehicula tellus. Fusce dui diam, aliquet sit amet eros at, sollicitudin facilisis quam. Phasellus tempor metus vel augue gravida pretium. Proin aliquam aliquam blandit. Nulla id tempus mi. Fusce in aliquam tortor.
% 		\\ \hline
		
% 		\emph{Pellentesque}
% 		& Donec felis nibh, imperdiet a arcu non, vehicula gravida nibh. Quisque interdum sapien eu massa commodo, ac elementum felis faucibus.
% 		\\ \hline
		
% 		\emph{Molestie}
% 		& Cras ullamcorper tellus et auctor ultricies. Maecenas tincidunt euismod lectus nec venenatis. Suspendisse potenti. Pellentesque pretium nunc ut euismod cursus. Nam venenatis condimentum quam. Curabitur suscipit efficitur aliquet. Interdum et malesuada fames ac ante ipsum primis in faucibus.
% 		\\ \hline
		
% 		\emph{Vivamus semper}
% 		& In purus purus, faucibus eu libero vulputate, tristique sodales nunc. Nulla ut gravida dolor. Fusce vel pellentesque mi, vel efficitur eros. Nunc vitae elit tellus. Sed vestibulum auctor consequat. 
% 		\\ \hline
		
% 		\emph{Condimentum}
% 		& Nulla scelerisque, leo et facilisis pretium, risus enim cursus turpis, eu suscipit ipsum ipsum in mauris. Praesent eget pulvinar ipsum, suscipit interdum nunc. Nam varius massa ut justo ullamcorper sollicitudin. Vivamus facilisis suscipit neque, eu fermentum risus. Ut at mi mauris.
% 		\\ \hline
		
% 		\caption{Praesent ullamcorper consequat tellus ut eleifend}
% 		\label{tab:example-3}		
% 	\end{longtable}
% \end{center}
