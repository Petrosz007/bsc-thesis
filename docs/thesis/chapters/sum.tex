\chapter{Összegzés} % Conclusion
\label{ch:sum}

Összességében szerintem sikerült specifikációnak megfelelően implementálnom az alkalmazást. Nem először csináltam ilyen architectúrával fullstack webes alkalmazást, így az alapok megtanulása helyett a jó architektúrális döntésekbe fektethettem sok időt, így egy könnyen bővíthető, könnyen skálázható programot tudtam létrehozni.

A szoftverfejlesztési folyamatot is próbáltam minél mondernebb eszközökkel könnyíteni. A CI/CD és jó teszt lefedettség segítségével hamar kijöttek fejlesztés közbeni hibák. Docker konténerkkel egyszerűen és platform függetlenül deploy-olható lett az alkalmazás, egyszerű konfogurációval. Git és Github miatt a kód verziókövetve és a felhőben biztonságosan volt tárolva.

Az egyetemen tanult tudásomat is a lehető legjobban kihasználtam. Webes Alkalmazások Fejlesztésétől, Bevezetés a DevOps tárgyon át a Funkcionális programozás és Haladó Haskell órák anyagait mind hasznosítani tudtam. Ezeken kívül az egyetem mellett tanult technológiáknak is nagy szerepe volt a projektben, Typescriptet és React.js-t magamtól a szabadidőmben tanultam, most ezt tudtam kamatoztatni.

Szeretnék köszönetet nyilvánítani Gyimesi Kristófnak, Szalai Patriknak és Hajdu Marcell-nek, amiért egész félévben kíméletlenül bombázhattam őket kérdéseimmel, és segítettek a véleményükkel, hogy a lehető legjobb programot tudjam kiadni a kezeim közül.

\clearpage

\section{További fejlesztői lehetőségek}
Az Időpont foglaló webes alkalmazás már így is nagyon sok funkcióval bír, viszont, mint egy jó alkalmazás, soha sem állhat meg a fejlődésben. A következő jövőbeli fejlesztési ötletek kivitelezésében gondolkom:

\begin{compactitem}
    \item Email service, ami regisztrációnál megerősíti a fiókot, jelszó emlékeztetőt tud küldeni, esetleg időpont változásnál értesítést
    \item Mobilra optimalizált frontend
    \item Mobil optimalizált alkalmazások. Akár Progresszív Web Applikáció, akár natív pl.: React Native keretrendszerrel
    \item Adószám, számlázási infó, hogy lehessen rendes tényleges számlát kiállítani, esetleg fizetési rendszert integrálni, akár Számlázz.hu-n keresztül
    \item Lemondás X időn (pl.: 24 órán) belül nem lehetséges, ezt kategóriánként lehetne megadni
    \item Email címet használni felhasználónév helyett az alkalmazásban, mellette egy egyedi felhasználói azonosító
    \item CD, Kubernetes deployment, Selenium automatizált end-to-end tesztek
\end{compactitem}


